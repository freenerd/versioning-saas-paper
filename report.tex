
%%%%%%%%%%%%%%%%%%%%%%% file typeinst.tex %%%%%%%%%%%%%%%%%%%%%%%%%
%
% This is the LaTeX source for the instructions to authors using
% the LaTeX document class 'llncs.cls' for contributions to
% the Lecture Notes in Computer Sciences series.
% http://www.springer.com/lncs       Springer Heidelberg 2006/05/04
%
% It may be used as a template for your own input - copy it
% to a new file with a new name and use it as the basis
% for your article.
%
% NB: the document class 'llncs' has its own and detailed documentation, see
% ftp://ftp.springer.de/data/pubftp/pub/tex/latex/llncs/latex2e/llncsdoc.pdf
%
%%%%%%%%%%%%%%%%%%%%%%%%%%%%%%%%%%%%%%%%%%%%%%%%%%%%%%%%%%%%%%%%%%%


\documentclass[runningheads,a4paper]{llncs}

\usepackage{amssymb}
\setcounter{tocdepth}{3}
\usepackage{graphicx}

\usepackage{url}
\urldef{\mailsa}\path|{alfred.hofmann, ursula.barth, ingrid.haas, frank.holzwarth,|
\urldef{\mailsb}\path|anna.kramer, leonie.kunz, christine.reiss, nicole.sator,|
\urldef{\mailsc}\path|erika.siebert-cole, peter.strasser, lncs}@springer.com|
\newcommand{\keywords}[1]{\par\addvspace\baselineskip
\noindent\keywordname\enspace\ignorespaces#1}

\begin{document}

\mainmatter  % start of an individual contribution

% first the title is needed
\title{Versioning for Software as a Service in the context of Multi-Tenancy}

% a short form should be given in case it is too long for the running head

% the name(s) of the author(s) follow(s) next
%
% NB: Chinese authors should write their first names(s) in front of
% their surnames. This ensures that the names appear correctly in
% the running heads and the author index.
%
\author{Maximilian Schneider \and Johan Uhle}
%
% (feature abused for this document to repeat the title also on left hand pages)

% the affiliations are given next; don't give your e-mail address
% unless you accept that it will be published
\institute{University of Potsdam, Hasso-Plattner-Institute\\
Prof.-Dr.-Helmert-Str. 2-3, 14482 Potsdam, Germany\\
\path|{maximilian.schneider, johan.uhle}@student.hpi.uni-potsdam.de|}

%
% NB: a more complex sample for affiliations and the mapping to the
% corresponding authors can be found in the file "llncs.dem"
% (search for the string "\mainmatter" where a contribution starts).
% "llncs.dem" accompanies the document class "llncs.cls".
%

\toctitle{Lecture Notes in Computer Science}
\tocauthor{Authors' Instructions}
\maketitle


\begin{abstract}
The abstract should summarize the contents of the paper and should
contain at least 70 and at most 150 words. It should be written using the
\emph{abstract} environment.
\keywords{We would like to encourage you to list your keywords within
the abstract section}
\end{abstract}

\section{Introduction}

Boom \cite{Bezemer2010}

You are strongly encouraged to use \LaTeXe{} for the
preparation of your camera-ready manuscript together with the
corresponding Springer class file \verb+llncs.cls+. Only if you use
\LaTeXe{} can hyperlinks be generated in the online version
of your manuscript.

The \LaTeX{} source of this instruction file for \LaTeX{} users may be
used as a template. This is
located in the ``authors'' subdirectory in
\url{ftp://ftp.springer.de/pub/tex/latex/llncs/latex2e/instruct/} and
entitled \texttt{typeinst.tex}. There is a separate package for Word
users. Kindly send the final and checked source
and PDF files of your paper to the Contact Volume Editor. This is
usually one of the organizers of the conference. You should make sure
that the \LaTeX{} and the PDF files are identical and correct and that
only one version of your paper is sent. It is not possible to update
files at a later stage. Please note that we do not need the printed
paper.

We would like to draw your attention to the fact that it is not possible
to modify a paper in any way, once it has been published. This applies
to both the printed book and the online version of the publication.
Every detail, including the order of the names of the authors, should
be checked before the paper is sent to the Volume Editors.

\subsection{Checking the PDF File}

Kindly assure that the Contact Volume Editor is given the name and email
address of the contact author for your paper. The Contact Volume Editor
uses these details to compile a list for our production department at
SPS in India. Once the files have been worked upon, SPS sends a copy of
the final pdf of each paper to its contact author. The contact author is
asked to check through the final pdf to make sure that no errors have
crept in during the transfer or preparation of the files. This should
not be seen as an opportunity to update or copyedit the papers, which is
not possible due to time constraints. Only errors introduced during the
preparation of the files will be corrected.

This round of checking takes place about two weeks after the files have
been sent to the Editorial by the Contact Volume Editor, i.e., roughly
seven weeks before the start of the conference for conference
proceedings, or seven weeks before the volume leaves the printer's, for
post-proceedings. If SPS does not receive a reply from a particular
contact author, within the timeframe given, then it is presumed that the
author has found no errors in the paper. The tight publication schedule
of LNCS does not allow SPS to send reminders or search for alternative
email addresses on the Internet.

In some cases, it is the Contact Volume Editor that checks all the final
pdfs. In such cases, the authors are not involved in the checking phase.

\subsection{Additional Information Required by the Volume Editor}

If you have more than one surname, please make sure that the Volume Editor
knows how you are to be listed in the author index.

\subsection{Copyright Forms}

The copyright form may be downloaded from the ``For Authors"
(Information for LNCS Authors) section of the LNCS Website:
\texttt{www.springer.com/lncs}. Please send your signed copyright form
to the Contact Volume Editor, either as a scanned pdf or by fax or by
courier. One author may sign on behalf of all of the other authors of a
particular paper. Digital signatures are acceptable.


\bibliography{./bib/SaaS-paper}{}
\bibliographystyle{abbrvnat}

\end{document}
