\section{Introduction (Max)}

The recent trend of providing Software as a Service instead of via the classical retail channels has changed the way that this software is versioned.
Earlier software was released infrequently and the customer would choose when to switch to a different version after it was released.
Nowadays a Software as a Service provider is able to release updates to the provided software continuosly and deploy them without further action of the customer aswell.
\\
Improving upon that providers may even choose to provide multiple versions at the same time in order to compare these.
For instance changes to the user-interface are tried out in A/B split tests especially by web companies.
In this way they are able to decide whether a certain change is an actual improvement based on the behaviour of the different split groups.
Also new features which might result in a considerable performance degredation can be tried out on a subset of all users in order to verify that they are actually feasible in production.

-- aber mehr versionen != geiler
--   schwerer festzustellen



In that way the customer could decide to delay upgrading, when switching the version might impair their usage of the software.

What is SaaS and multi-tenancy and why is it important? (better explained in \ref{sec:terminology}?)

What is versioning?

Reasons for versioning
  Manage incompatibility
    Feature change (User acceptance)
    API compatibility (Technical incompatibility)

  Rollout of features
    Ops: Does it scale?
    Biz/Users: Is it accepted by users? Does the feature 'work' (A/B testing)
  - Legal

  + provider pov
  needs flexibility to react to changed requirements
  reduce technical debt by retiring code/functionality

  + user pov
  client will complain and will want to stay on the old version
