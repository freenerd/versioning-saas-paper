%!TEX root = ../report.tex

\section{Evaluation}
\label{sec:eval}

We evaluate our findings of the previous sections with our experience in developing SaaS systems ourselves.

\paragraph{Project A} was a multi-user SaaS with over 10 million registered users. It served a website as well as a REST API from a Ruby on Rails web application backed by MySQL servers. At one point it was decided to build a new version of the website, mostly to implement fundamental changes in the user interface. Also old features were removed or fundamentally changed as well as new features added. A team was elected to work on the new website, while another team continued to maintain the old website. Furthermore it was decided to build the new version as an API client on top of the current REST API. A new code base with own deployment independent of the current website was created. During development it showed that the REST API did not support all features needed by the new version, thus the REST API was extended to support new endpoints. Also some schema changes were necessary, but they were either additions to existing tables or new tables. The Ruby on Rails application used Active Record as an ORM to access the database. Active Record handles additions to the schema in a forward-compatible way. Thus the schema changes needed for the new website were 100\% backwards compatible with the old website.

Once deployed, the new version was gradually rolled out the the users. Users were able to opt-in to the new version, but also to switch back to the old version. Anonymous users were initially always served the old version, but once the new version was officially launched all anonymous users got served the new version. The respective version per user was saved in a cookie. When a user requested the website, the web server receiving the request decided based on the cookie information, to which application server the request should be routed.

The versioning in this example happened outside of our architecture proposals in the previous Section~\ref{sec:architecture}, since the new version became an API client of the old version. It still relates to the shared-instance 1:1 approach from Section~\ref{sec:backend11}, since it has application servers per version and a routing layer routing requests based on versions. Also one of the insights it validates is that schema changes are possible between versions if they happen in a compatible way.

\paragraph{Project B} Max

